\documentclass{article}
\usepackage{arxiv}
\usepackage[utf8]{inputenc}
\usepackage[english, russian]{babel}
\usepackage[T1]{fontenc}
\usepackage{url}
\usepackage{booktabs}
\usepackage{amsfonts}
\usepackage{nicefrac}
\usepackage{microtype}
\usepackage{lipsum}
\usepackage{graphicx}
\usepackage{natbib}
\usepackage{doi}



\title{Классификация товаров по ОКПД2 кодам}

\author{ Фирсов Сергей \\
        Кафедра интеллектуальных систем\\
	МФТИ\\
	\texttt{firsov.sa@phystech.edu} \\
	\And
	Всеволод Михайлович Старожилец \\
	Кафедра интеллектуальных систем\\
	Форексис\\
	\texttt{vsevolod.starozhilets@antirutina.net} \\
	%% \AND
	%% Coauthor \\
	%% Affiliation \\
	%% Address \\
	%% \texttt{email} \\
	%% \And
	%% Coauthor \\
	%% Affiliation \\
	%% Address \\
	%% \texttt{email} \\
	%% \And
	%% Coauthor \\
	%% Affiliation \\
	%% Address \\
	%% \texttt{email} \\
}
\date{}

\renewcommand{\shorttitle}{\textit{arXiv} Template}

%%% Add PDF metadata to help others organize their library
%%% Once the PDF is generated, you can check the metadata with
%%% $ pdfinfo template.pdf
\hypersetup{
pdftitle={Декодирования сигналов головного мозга в аудиоданные},
pdfsubject={q-bio.NC, q-bio.QM},
pdfauthor={Набиев Мухаммадшариф, Северилов Павел},
pdfkeywords={First keyword, Second keyword, More},
}

\begin{document}
\maketitle

\begin{abstract}
    Исследование направлено на решение задачи классификации товаров по ОКПД 2 кодам с использованием кратких текстовых описаний. Коды представляют собой детализированную систему категоризации продуктов и услуг по видам экономической деятельности. Основная цель - повышение точности и сокращение ресурсозатратности классификации, анализируя влияние глубины классификатора ОКПД 2. Для достижения этих целей предлагается метод построения текстовых эмбеддингов с использованием нейросетевых технологий, таких как spaCy. Задача усложняется необходимостью предварительной обработки данных для перевода исходных описаний в стандартизированные короткие тексты, адаптированные для анализа. Используются данные государственных закупок по ФЗ 44 за 2022 год, охватывающие около 40\% открытых источников, что обеспечивает достаточный объем и разнообразие информации для анализа. Также часть этих данных будет критерием оценки работы программы. Новизна заключается в применении методов машинного обучения к индустриальной задаче, что обещает улучшение в процессах логистики, учёте и анализе в сфере закупок. 


 
\end{abstract}


\keywords{OKPD 2 code \and text analysis \and task of classification}

\section{Введение}

Целью данного исследования является разработка и апробация метода классификации товаров по ОКПД 2, используя краткие текстовые описания. Актуальность задачи обусловлена необходимостью повышения эффективности процессов логистики и учета в сфере закупок, а также сокращения времени и ресурсов, затрачиваемых на классификацию товаров.

Объектом исследования выступают любые товары, для которых возможна классификация по ОКПД 2 кодам (детализированной системе категоризации продукции и услуг по видам экономической деятельности). Проблема заключается в разработке метода, позволяющего автоматизировать этот процесс с высокой точностью и полнотой классификации, устойчиво относительно формата входных данных, и в исследовании характеристик этого метода (по указанным параметрам) от глубины классификации.

Задача классификации разобрана вдоль и поперёк в любой доступной литературе по машинному обучению. В дальнейшей работе будем опираться на курс лекций Воронцова К.В. и книги \cite{Goodfellow2016DeepLearning} и \cite{Montani2019AdvancedNLP}. Текстовые эмбеддинги тоже активно используются в современной разработке, есть много предобученных моделей и пакетов, таких как Word2Vec, GloVe, spaCy, по ним также найдены и частично изучены книги и современные результаты. 

Предлагаемое решение базируется на использовании передовых методов обработки естественного языка и машинного обучения: получения словаря слов-признаков на основе данных госзакупок, построение эмбендингов по ним, с использованием spaCy и решение задачи классификации по полученным векторам на основе результатов обучающей выборки.
Решение ново настолько, насколько нов подход к решению индустриальной задачи с помощью методов машинного обучения. Преимуществами такого подхода являются увеличение точности и снижение затрат времени на обработку данных, как раз то что требуется в прикладных задачах.

Цель эксперимента – проверка эффективности предложенного метода на реальных данных государственных закупок за 2022 год. Это позволяет оценить работу алгоритма в условиях большого объема и разнообразия данных. Экспериментальная установка включает в себя подготовку данных, построение модели классификации и ее тестирование с целью определения оптимальных параметров для достижения максимальной точности классификации в зависимости от необходимой глубины класификации. Рабочий процесс описывает последовательные шаги от предварительной обработки текстов до оценки результатов классификации.

В заключение, данное исследование представляет собой вклад в развитие методов машинного обучения и их применение к решению практических задач классификации товаров, что имеет важное значение для сферы государственных закупок и управления цепочками поставок.

\section{Постановка проблемы}
Sample set (X) - данные госзакупок за 2022 год. Представляют собой таблицу из двух колонок, в первой - краткое (почти наверное) описание товара, во второй - заданный ОКПД 2 код этого товара. Выборка из более 8 миллионов результатов, считается полностью корректной, с точностью до описания товаров. Записи могут частично повторяться или незначительно отличаться текстовым описанием. 

Будет использоваться линейная модель классификации, $A = \{ \ g(x,\theta) \ | \ \theta \in \mathcal{R} \ \}$ , где $g(x, \theta) = sign \sum \limits_{j=1}^n \theta_j f_j(x)$.

Используется квадратичная функция потерь: $\mathcal{L}(a,x) = (a(x) - y(x))^2$, где $y$ - значения контрольной выборки.  

Откуда получаем функцию Эмпирического риска : $Q(a,X^l)=\frac{1}{l}\sum\limits^l_{i=1}\mathcal{L}(a,x_i)$

И будем решать задачу оптимизации - минимизации эмпирического риска 
$$ \mu(X^l) = arg \min_{a\in A} Q(a, X^l)$$


Критерий качество в задаче нашей классификации предельно ясен - попадание конкретного товар в свою категорию и в правильный ОКПД 2 код.

\bibliographystyle{unsrtnat}
\bibliography{Firsov2024Classification_according_to_OKPD_2_codes}

\end{document}