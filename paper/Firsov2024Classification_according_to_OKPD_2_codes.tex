\documentclass{article}
\usepackage{arxiv}
\usepackage[utf8]{inputenc}
\usepackage[english, russian]{babel}
\usepackage[T1]{fontenc}
\usepackage{url}
\usepackage{booktabs}
\usepackage{amsfonts}
\usepackage{nicefrac}
\usepackage{microtype}
\usepackage{lipsum}
\usepackage{graphicx}
\usepackage{natbib}
\usepackage{doi}



\title{Классификация товаров по ОКПД2 кодам}

\author{ Фирсов Сергей \\
        Кафедра интеллектуальных систем\\
	МФТИ\\
	\texttt{firsov.sa@phystech.edu} \\
	\And
	Всеволод Михайлович Старожилец \\
	Кафедра интеллектуальных систем\\
	Форексис\\
	\texttt{vsevolod.starozhilets@antirutina.net} \\
	%% \AND
	%% Coauthor \\
	%% Affiliation \\
	%% Address \\
	%% \texttt{email} \\
	%% \And
	%% Coauthor \\
	%% Affiliation \\
	%% Address \\
	%% \texttt{email} \\
	%% \And
	%% Coauthor \\
	%% Affiliation \\
	%% Address \\
	%% \texttt{email} \\
}
\date{}

\renewcommand{\shorttitle}{\textit{arXiv} Template}

%%% Add PDF metadata to help others organize their library
%%% Once the PDF is generated, you can check the metadata with
%%% $ pdfinfo template.pdf
\hypersetup{
pdftitle={Декодирования сигналов головного мозга в аудиоданные},
pdfsubject={q-bio.NC, q-bio.QM},
pdfauthor={Набиев Мухаммадшариф, Северилов Павел},
pdfkeywords={First keyword, Second keyword, More},
}

\begin{document}
\maketitle

\begin{abstract}
    Исследование направлено на решение задачи классификации товаров по ОКПД 2 кодам с использованием кратких текстовых описаний. Основная цель - повышение точности и сокращение ресурсозатратности классификации, анализируя влияние глубины классификатора ОКПД 2. Для достижения этих целей предлагается метод построения текстовых эмбеддингов с использованием нейросетевых технологий, таких как spaCy. Задача усложняется необходимостью предварительной обработки данных для трансформации исходных описаний в стандартизированные короткие тексты, адаптированные для анализа. Используются данные государственных закупок 2022 года, охватывающие около 40\% открытых источников, что обеспечивает достаточный объем и разнообразие информации для анализа. Также часть этих данных будет критерием оценки работы программы. Новизна заключается в применении методов машинного обучения к индустриальной задаче, что обещает улучшение в процессах логистики, учёте и анализе в сфере закупок.


 
\end{abstract}


\keywords{OKPD 2 code \and text analysis \and task of classification}

\section{Введение}



\bibliographystyle{unsrtnat}
\bibliography{references}

\end{document}